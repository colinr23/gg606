% Options for packages loaded elsewhere
\PassOptionsToPackage{unicode}{hyperref}
\PassOptionsToPackage{hyphens}{url}
%
\documentclass[
]{book}
\usepackage{lmodern}
\usepackage{amssymb,amsmath}
\usepackage{ifxetex,ifluatex}
\ifnum 0\ifxetex 1\fi\ifluatex 1\fi=0 % if pdftex
  \usepackage[T1]{fontenc}
  \usepackage[utf8]{inputenc}
  \usepackage{textcomp} % provide euro and other symbols
\else % if luatex or xetex
  \usepackage{unicode-math}
  \defaultfontfeatures{Scale=MatchLowercase}
  \defaultfontfeatures[\rmfamily]{Ligatures=TeX,Scale=1}
\fi
% Use upquote if available, for straight quotes in verbatim environments
\IfFileExists{upquote.sty}{\usepackage{upquote}}{}
\IfFileExists{microtype.sty}{% use microtype if available
  \usepackage[]{microtype}
  \UseMicrotypeSet[protrusion]{basicmath} % disable protrusion for tt fonts
}{}
\makeatletter
\@ifundefined{KOMAClassName}{% if non-KOMA class
  \IfFileExists{parskip.sty}{%
    \usepackage{parskip}
  }{% else
    \setlength{\parindent}{0pt}
    \setlength{\parskip}{6pt plus 2pt minus 1pt}}
}{% if KOMA class
  \KOMAoptions{parskip=half}}
\makeatother
\usepackage{xcolor}
\IfFileExists{xurl.sty}{\usepackage{xurl}}{} % add URL line breaks if available
\IfFileExists{bookmark.sty}{\usepackage{bookmark}}{\usepackage{hyperref}}
\hypersetup{
  pdftitle={Scientific Data Wrangling},
  pdfauthor={Colin Robertson},
  hidelinks,
  pdfcreator={LaTeX via pandoc}}
\urlstyle{same} % disable monospaced font for URLs
\usepackage{longtable,booktabs}
% Correct order of tables after \paragraph or \subparagraph
\usepackage{etoolbox}
\makeatletter
\patchcmd\longtable{\par}{\if@noskipsec\mbox{}\fi\par}{}{}
\makeatother
% Allow footnotes in longtable head/foot
\IfFileExists{footnotehyper.sty}{\usepackage{footnotehyper}}{\usepackage{footnote}}
\makesavenoteenv{longtable}
\usepackage{graphicx}
\makeatletter
\def\maxwidth{\ifdim\Gin@nat@width>\linewidth\linewidth\else\Gin@nat@width\fi}
\def\maxheight{\ifdim\Gin@nat@height>\textheight\textheight\else\Gin@nat@height\fi}
\makeatother
% Scale images if necessary, so that they will not overflow the page
% margins by default, and it is still possible to overwrite the defaults
% using explicit options in \includegraphics[width, height, ...]{}
\setkeys{Gin}{width=\maxwidth,height=\maxheight,keepaspectratio}
% Set default figure placement to htbp
\makeatletter
\def\fps@figure{htbp}
\makeatother
\setlength{\emergencystretch}{3em} % prevent overfull lines
\providecommand{\tightlist}{%
  \setlength{\itemsep}{0pt}\setlength{\parskip}{0pt}}
\setcounter{secnumdepth}{5}
\usepackage{booktabs}
\ifluatex
  \usepackage{selnolig}  % disable illegal ligatures
\fi
\usepackage[]{natbib}
\bibliographystyle{apalike}

\title{Scientific Data Wrangling}
\author{Colin Robertson}
\date{2021-11-16}

\begin{document}
\maketitle

{
\setcounter{tocdepth}{1}
\tableofcontents
}
\hypertarget{prerequisites}{%
\chapter{prerequisites}\label{prerequisites}}

Moving from data acquired by a sensor or in the field to a model or visualization that can provide insights to a question often requires an extensive amount of work. It is estimated that `data wrangling' - cleaning, loading, processing, integrating their data comprises at \href{https://www.datanami.com/2020/07/06/data-prep-still-dominates-data-scientists-time-survey-finds/}{least half of a data scientist's time}, and that may be even higher in the context of environemental data science.

\hypertarget{textbook}{%
\section{textbook}\label{textbook}}

Wickham H, Grolemund G. 2017. R for Data Science: Import, Tidy, Transform, Visualize, and Model Data. O'Reilly Media. Chicago, available online at \url{http://r4ds.had.co.nz/}

\hypertarget{additional-readings}{%
\section{additional readings}\label{additional-readings}}

Broman KW, Woo KH. 2018. Data Organization in Spreadsheets. The American Statistician 72(1): 2-10. \url{https://doi.org/10.1080/00031305.2017.1375989}

Bryan J, et al.~2018. Happy Git and GitHub for the useR. \url{http://happygitwithr.com/}

Hampton SE, Anderson SS, Bagby SC, Gries C, Han X, Hart EM, Jones MB, Lenhardt WC, MacDonald A, Michener WK, Mudge J, Pourmokhtarian A, Schildhauer MP, Woo KH, Zimmerman N. 2015. The Tao of open science for ecology. Ecosphere 6(7):120. \url{http://dx.doi.org/10.1890/ES14-00402.1}

Hart EM, Barmby P, LeBauer D, Michonneau F, Mount S, Mulrooney P, et al.~2016. Ten Simple Rules for Digital Data Storage. PLoS Comput Biol12(10): e1005097. \url{https://doi.org/10.1371/journal.pcbi.1005097}

Sixteen peer-reviewed journal articles in the PeerJ Collection, Practical Data Science for Stats: \url{https://peerj.com/collections/50-practicaldatascistats/}

Wilke CO. 2019. Fundamental of Data Visualization: A Primer on Making Informative and Compelling Figures. O'Reilly Media. Chicago, \url{https://serialmentor.com/dataviz/}

\hypertarget{schedule}{%
\section{schedule}\label{schedule}}

Wrangling

Week

Topic

Tools

\begin{enumerate}
\def\labelenumi{\arabic{enumi}.}
\item
  Data wrangling

  R (base)

  \begin{enumerate}
  \def\labelenumii{\arabic{enumii}.}
  \setcounter{enumii}{1}
  \item
    i/o

    R (base, rgdal)

    \begin{enumerate}
    \def\labelenumiii{\arabic{enumiii}.}
    \setcounter{enumiii}{2}
    \item
      data objects

      R (tidyverse)

      \begin{enumerate}
      \def\labelenumiv{\arabic{enumiv}.}
      \setcounter{enumiv}{3}
      \item
        databases

        R (tidyverse, dbply), SQLite

        \begin{enumerate}
        \def\labelenumv{\arabic{enumv}.}
        \item
          spatial data

          R (rgdal, sp, sf)

          \begin{enumerate}
          \def\labelenumvi{\arabic{enumvi}.}
          \item
            spatial data

            R (spatstat, gstat)

            \begin{enumerate}
            \def\labelenumvii{\arabic{enumvii}.}
            \item
              temporal data

              R (ts, zoo, lubridate
            \end{enumerate}
          \end{enumerate}
        \end{enumerate}
      \end{enumerate}
    \end{enumerate}
  \end{enumerate}
\end{enumerate}

Workflows

Week

Topic

Tools

\begin{enumerate}
\def\labelenumi{\arabic{enumi}.}
\setcounter{enumi}{7}
\item
  reproducible research

  Git, Github

  \begin{enumerate}
  \def\labelenumii{\arabic{enumii}.}
  \setcounter{enumii}{8}
  \item
    data sharing and collaboration

    r markdown, docker

    \begin{enumerate}
    \def\labelenumiii{\arabic{enumiii}.}
    \setcounter{enumiii}{9}
    \item
      package development

      tbd
    \end{enumerate}
  \end{enumerate}
\end{enumerate}

Project management

Week

Topic

Tools

\begin{enumerate}
\def\labelenumi{\arabic{enumi}.}
\setcounter{enumi}{10}
\item
  R in Production

  R

  \begin{enumerate}
  \def\labelenumii{\arabic{enumii}.}
  \setcounter{enumii}{11}
  \item
    Presentations

    n/a
  \end{enumerate}
\end{enumerate}

\hypertarget{evaluation}{%
\section{evaluation}\label{evaluation}}

Practical Work

There will be two assignments and one term project required for the course. Students will participate in an analytics studio workshop which will involve critically analyzing projects. Coursework will involve the use of a variety GIS and/or statistical software packages.

\hypertarget{grading}{%
\subsection{grading}\label{grading}}

\begin{longtable}[]{@{}lc@{}}
\toprule
practical work & \%\tabularnewline
\midrule
\endhead
Two term assignments & 30\%\tabularnewline
Term project write-up & 40\%\tabularnewline
Analytics studio workshop & 15\%\tabularnewline
Class participation & 15\%\tabularnewline
Total: & 100\%\tabularnewline
\bottomrule
\end{longtable}

\hypertarget{intro}{%
\chapter{introduction}\label{intro}}

\hypertarget{what-is-spatial-knowledge-mobilization}{%
\section{What is Spatial Knowledge Mobilization?}\label{what-is-spatial-knowledge-mobilization}}

\hypertarget{forms-of-spatial-knowledge}{%
\subsection{Forms of Spatial Knowledge}\label{forms-of-spatial-knowledge}}

\hypertarget{knowledge-mobilization}{%
\subsection{Knowledge Mobilization}\label{knowledge-mobilization}}

\hypertarget{data-io}{%
\chapter{data io}\label{data-io}}

A list of topics to be covered in this section;

\begin{itemize}
\tightlist
\item
  geospatial data formats

  \begin{itemize}
  \tightlist
  \item
    vector and OGC
  \item
    raster
  \end{itemize}
\item
  rgdal
\item
  netCDF
\item
  stacs
\item
  API access
\item
  IoT and sensor data
\end{itemize}

\hypertarget{cloud-geo}{%
\chapter{cloud geo}\label{cloud-geo}}

The preceding chapter assumes a workflow where you are bringing data from elsewhere to your local machine and to perform further analysis. Many platforms now exist to support alternate workflows, where data and analysis remain on the cloud. The space of cloud geospatial is rapidly evolving. In this section we will review some of the more common platforms, discuss strengths and weaknesses, and when you might want to incorporate cloud-based workflows into EDA projects.

\hypertarget{imagery-data}{%
\chapter{imagery data}\label{imagery-data}}

Geospatial imagery is an increasingly important type of information source for EDA applications. We consider any image data where earth/environment is captured on pixel arrays and used to better understand environmental phenomena. Such a broad approach to geospatial imagery captures tradtional sources such as earth-observation satellite-based sensors as well as non-traditional platforms such as smartphone cameras, RPS-based image sources, and handheld imaging platforms. Depending on how geospatial information is integrated into image data, there are different considerations for how you might integrated them within a project. We will review some common issues and techniques for handling a variety of image data sources in this chapter.

  \bibliography{book.bib,packages.bib}

\end{document}
